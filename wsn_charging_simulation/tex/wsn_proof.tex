\usepackage{amsfonts}

\documentstyle[11pt]{article}
\begin{document}
\title{NP-Complete Proof - UAV Wireless Power Transfer for Wireless Sensor Network}
\author{Jinfu Leng}
\maketitle
\section{Background}
UAV can be used to charge nodes in wireless sensor network to prolong the life of the network. To charge a node, UAV has to fly to the specific node. UAV itself can be charged at UAV base station. To be charged, UAV has to fly to the UAV base station. UAV starts from UAV base station, and it costs equal amount of resources when the UAV flies out of the UAV base station. Given the information of the UAV, UAV base station and network, what is the optimal strategy for UAV to keep the network alive in the given time limit with minimal cost of resources.

\section{Description}
\textbf{Instance} 
A graph $G = (V, E)$, a vertex $b \in V$, a series of numbers $e_{max} \in \mathbb{R}_{>0}$, $e_{init} \in [0, e_{max}]$, $e_{cf} \in \mathbb{R}_{>0}$, $e_{ct} \in \mathbb{R}_{>0}$, $r \in [0, 1]$, $e_r \in [0, e_{max}]$, $v \in \mathbb{R}_{>0}$, $E_{max} \in \mathbb{R}_{>0}$, $E_{init} \in [0, E_{max}]$, $E_{min} \in \mathbb{R}_{\geq 0}$ and $E_{c} \in \mathbb{R}_{>0}$.

\textbf{Question} 
One UAV, has energy capacity of $e_{max}$ and initial energy of $e_{init}$. UAV consumes energy of $e_{cf}$ per second for flight, and energy of $e_{ct}$ per second for transferring energy from UAV to the nodes in the network. The transfer efficiency rate is $r$ (i.e. node can receive energy of $e_{ct}*r$ per second). When UAV is at UAV base station, UAV can receive energy of $e_r$ per second from UAV base station. The moving distance of UAV is $v$ per second.

Set
We consider $N$ sensor nodes and a UAV base station are distributed over a graph $G_s$. Each sensor node has the battery capacity of $E_{max}$, the initial energy of $E_{init}$, and the minimal energy of $E_{min}$. We say that the network is dead when the energy of any of the nodes falls below $E_{min}$. 

There is an UAV in the system to charge these nodes. .   . Initially, UAV is at the base station, and UAV has to come back to the base station before its power falls below $0$. The moving speed of UAV is $v$.

For UAV, there are two types of valid commands:
1, fly from node $A$ to node $B$;
2, charge node $A$ for $t$ seconds;

The problem is, suppose we want to keep the network to work as long as time of $T$, what is the minimal number of charging tasks the UAV has to execute. Every time UAV going out of the base station counts as one charging task.

The decision version of the problem is, given time of $T$, is the UAV able to keep the system working as long as $T$ with at most $k$ tasks.
\section{Proof}

First, the problem is in $NP$.

Given a sequence of UAV actions $S$, we can simulate the actions and verify it. The time complexity is $O(|S| * |N|)$.

Second, given an instance of the decision version of $Travelling Salesman Problem$, $G = (V, E)$ and $L$, we can transform it into an instance of $UAV Charging Problem$ in polynomial time.

We can construct $G_s = G$, and then randomly choose a vertex as the UAV base station. All the other $|V|-1$ vertices are wireless sensor network nodes.

We set the parameters of nodes as: $E_{max} = |V| + |L| - 1$, $E_{init} = |V| + |L| - 2$, $E_{min} = 0$, and $E_{c} = 1$.

We set the parameters of UAV as: $e_{max} = |V| + |L| - 1$, $e_{init} = |V| + |L| - 1$, $e_{cf} = 1$, $e_{ct} = 1$, $r = 1.0$, $e_r = 0$, and $v = 1$.

Then we ask, is the UAV able to keep the system working as long as $|V| + |L| - 1$ with at most $1$ task? Because every node need to be charged one second, and there are $|V|-1$ nodes, the answer is yes if and only if the UAV is able to find a path of length less than $|L|$ to visit all the nodes and come back to the station.

Also, it is clear that this transformation can be finished in polynomial time.
\end{document}