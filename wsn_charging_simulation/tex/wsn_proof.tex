\documentclass[11pt]{article}
\usepackage{amssymb,amsmath}
\usepackage{enumitem}

\begin{document}
\title{UAV Wireless Power Transfer for Wireless Sensor Nodes - UAVWW}
\author{Jinfu Leng}
\maketitle
\section{Background}
UAV can be used to charge nodes to prolong the life of the wireless sensor network. To charge a node, UAV has to fly to the specific node. UAV itself can be charged at UAV base station. Given the information of the UAV, UAV base station and wireless sensor nodes, what is the optimal strategy for UAV to keep the network alive in a given time limit with minimal cost of resources? We call this problem as UAVWW problem.

\section{Problem Definition}
\textbf{NOTATIONS}

The notations of the problem are summarized as below:
\begin{itemize}[noitemsep]
\item $G_s = (V, E)$ and $G_s$ satisfies triangle inequality, the distribution of the sensor nodes and UAV base station
\item $V_{base} \in V$, UAV base station
\item $V_{nodes} = V - \{V_{base}\}$, sensor nodes
\item $E_{UAV\_max} \in \mathbb{R}_{\geq0}$, the maximal energy of the UAV
\item $E_{UAV\_init} \in [0, E_{UAV\_max}]$, the initial energy of the UAV
\item $e_{cf} \in \mathbb{R}_{>0}$, the energy consumption rate of UAV for flight
\item $e_{ct} \in \mathbb{R}_{>0}$, the energy consumption rate of UAV for transferring energy from UAV to sensor nodes
\item $r \in \mathbb{R}$ and $r \in [0.0, 1.0]$, the efficiency rate of transferring energy from UAV to sensor nodes
\item $e_r \in \mathbb{R}_{\geq0}$, the energy accumulation rate of UAV for receiving energy from UAV base station
\item $v \in \mathbb{R}_{\geq0}$, the moving speed of UAV
\item $E_{node\_max} \in \mathbb{R}_{\geq0}$, the maximal energy of the sensor nodes
\item $E_{node\_init} \in [0, E_{node\_max}]$, the initial energy of the sensor nodes
\item $e_{c} \in \mathbb{R}_{>0}$, the energy consumption rate of sensor node
\item $T \in \mathbb{R}_{\geq0}$, the required working time of the system
\item $k \in \mathbb{N}_{\geq0}$, the number of UAV tasks
\end{itemize}

\noindent
\textbf{Description}

UAV has four types of instructions:
\begin{enumerate}[noitemsep]
\item Hover at current position for time of $t$, $t \in \mathbb{R}_{>0}$
\item Fly from $A$ to $B$, $A,B \in V$ and UAV is at $A$
\item Transfer energy to $A$ for time of $t$, $A \in V_{nodes}$, $t \in \mathbb{R}_{>0}$ and UAV is at $A$
\item Charge UAV for time of $t$, $t \in \mathbb{R}_{>0}$ and UAV is at $V_{base}$
\end{enumerate}

We say the system is dead when any of the below conditions is satisfied:
\begin{enumerate}[noitemsep]
\item the energy of the UAV falls to 0 and UAV is not at $V_{base}$;
\item the energy of any of the sensor nodes falls to 0.
\end{enumerate}

The initial position of UAV is $V_{base}$. We count it as one UAV task when UAV flies out from $V_{base}$.

Given the information of the system ($G_s$, $V_{base}$, $V_{nodes}$, $E_{UAV\_max} $, $E_{UAV\_init}$, $e_{cf}$, $e_{ct}$, $r$, $e_r$, $v$, $E_{node\_max}$, $E_{node\_init}$, $e_{c}$), is there a finite sequence of valid UAV instructions can keep the system alive for time of $T$ with at most $k$ UAV task?

\section{Proof}
It is easy to see that UAVWW $\in NP$, because a nondeterministic algorithm need only guess a sequence of UAV instructions and check in polynomial time that there is no invalid instruction and the system keeps alive for more than time of $T$ while the number of UAV tasks is less than or equal to $k$.

We can transform METRIC-TSP to UAVWW. Let an arbitrary instance of METRIC-TSP be the graph $G = (V, E)$ and $L \in \mathbb{N}_{>=0}$. We can construct an instance of UAVWW as below. It is easy to see that this transformation can be finished in polynomial time.

\begin{tabular}{| l | l |}
\hline
$G_s$ & $G = (V, E)$ \\
$V_{base}$ & a random vertex from $V$ \\
$V_{nodes}$ & $V - \{V_{base}\}$ \\
$E_{UAV\_max}$ & $|V| + L - 1.0$ \\
$E_{UAV\_init}$ & $|V| + L - 1.0$ \\
$e_{cf}$ & $1.0$ \\
$e_{ct}$ & $1.0$ \\
$r$ & $1.0$ \\
$e_r$ & $0.0$ \\
$v$ & $1.0$ \\
$E_{node\_max}$ & $|V| + L - 1.0$ \\
$E_{node\_init}$ & $|V| + L - 2.0$ \\
$e_{c}$ & $1.0$ \\
$T$ & $|V| + L - 1.0$ \\
$k $ & $1$ \\
\hline
\end{tabular}

The idea of the transformation is to make an instance that the UAV has to visit and charge all the sensor nodes. At the same time, we set $k = 1$, so that the UAV has to visit all these sensor nodes and come back to UAV base station in a single tour. In addition, we limit the initial energy of the UAV so that the UAV can only flies for distance of at most $L$. In this construction, the answer of the transformed instance of UAVWW is yes if and only if the instance of METRIC-TSP has a hamiltonian cycle with length of at most $L$.

The detailed discussion of the transformed instance is showed below:
\begin{enumerate}[noitemsep]
\item The required time is $T = |V| + L - 1.0$, the initial energy of each node is $E_{node\_init} = |V| + L - 2.0$ and the consumption rate of each node is $e_{c} = 1.0$, so each node need to be charged at least $1.0$ unit of energy. \item The transfer efficiency rate is $r = 1.0$, so it requires energy of at least $|V| - 1.0$ for UAV to charge all these $|V| - 1$ nodes.
\item The number of tasks is $k = 1$, so UAV has to charge all the nodes in a single tour and then come back to UAV base station.
\item The initial energy of UAV is $E_{UAV\_init} = |V| + L - 1.0$ and it requires to transfer energy of at least $|V| - 1.0$ to nodes, so UAV has  energy of at most $L$ for flight. 
\item The moving speed is $v = 1.0$, the energy consumption rate of flying is  $e_{cf} = 1.0$ and UAV has energy of at most $L$ for flying, so UAV can fly distance of at most $L$.
\item UAV can move distance of at most $L$ and UAV has to visit all the nodes and then fly back to base station in a single tour, so we can claim that the transformed instance of UAVWW is yes if and only if the instance of METRIC-TSP has a satisfied cycle.
\end{enumerate}

\end{document}