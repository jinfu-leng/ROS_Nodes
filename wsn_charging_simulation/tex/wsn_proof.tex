\documentstyle[11pt]{article}
\begin{document}
\title{NP-Complete Proof - UAV Wireless Power Transfer for Wireless Sensor Network}
\author{Jinfu Leng}
\maketitle
\section{Background}
It is expensive to replace battery of nodes in wireless sensor network.
It is flexible and affordable to use UAV to charge nodes in wirelss sensor network.
\section{Description}
We consider $N$ sensor nodes and a UAV base station are distributed over a graph $G_s$. Each sensor node has the battery capacity of $E_{max}$, the initial energy of $E_{init}$, and the minimal energy of $E_{min}$. We say that the network is dead when the energy of any of the nodes falls below $E_{min}$. Each node consumes energy of $E_{c}$ per second.

There is an UAV in the system to charge these nodes. The UAV has the battery capacity of $e_{max}$. UAV consumes energy of $e_{cf}$ per second for flight, and consumes energy of $e_{ct}$ per second for transferring power to the wireless sensor nodes. The transfer rate from UAV to nodes is $r$. UAV can receive energy of $e_r$ per second from UAV base station. Initially, UAV is at the base station, and UAV has to come back to the base station before its power falls below $0$. The moving speed of UAV is $v$.

For UAV, there are two types of valid commands:
1, fly from node $A$ to node $B$;
2, charge node $A$ for $t$ seconds;

The problem is, suppose we want to keep the network to work as long as time of $T$, what is the minimal number of charging tasks the UAV has to execute. Every time UAV going out of the base station counts as one charging task.

The decision version of the problem is, given time of $T$, is the UAV able to keep the system working as long as $T$ with at most $k$ tasks.
\section{Proof}

First, the problem is in $NP$.

Given a sequence of UAV actions $S$, we can simulate the actions and verify it. The time complexity is $O(|S| * |N|)$.

Second, given an instance of the decision version of $Travelling Salesman Problem$, $G = (V, E)$ and $L$, we can transform it into an instance of $UAV Charging Problem$ in polynomial time.

We can construct $G_s = G$, and then randomly choose a vertex as the UAV base station. All the other $|V|-1$ vertices are wireless sensor network nodes.

We set the parameters of nodes as: $E_{max} = |V| + |L| - 1$, $E_{init} = |V| + |L| - 2$, $E_{min} = 0$, and $E_{c} = 1$.

We set the parameters of UAV as: $e_{max} = |V| + |L| - 1$, $e_{init} = |V| + |L| - 1$, $e_{cf} = 1$, $e_{ct} = 1$, $r = 1.0$, $e_r = 0$, and $v = 1$.

Then we ask, is the UAV able to keep the system working as long as $|V| + |L| - 1$ with at most $1$ task? Because every node need to be charged one second, and there are $|V|-1$ nodes, the answer is yes if and only if the UAV is able to find a path of length less than $|L|$ to visit all the nodes and come back to the station.

Also, it is clear that this transformation can be finished in polynomial time.
\end{document}