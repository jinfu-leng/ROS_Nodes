\documentclass[11pt]{article}
\usepackage{amssymb,amsmath}

\begin{document}
\title{NP-Complete Proof - UAV Wireless Power Transfer for Wireless Sensor Network}
\author{Jinfu Leng}
\maketitle
\section{Background}
UAV can be used to charge nodes in wireless sensor network to prolong the life of the network. To charge a node, UAV has to fly to the specific node. UAV itself can be charged at UAV base station. To be charged, UAV has to fly to the UAV base station. UAV starts from UAV base station, and it costs equal amount of resources when the UAV flies out of the UAV base station. Given the information of the UAV, UAV base station and network, what is the optimal strategy for UAV to keep the network alive in the given time limit with minimal cost of resources.

\section{Description}
\noindent
\textbf{INSTANCE:} 
A graph $G = (V, E)$, a vertex $base \in V$, a series of numbers $e_{max} \in \mathbb{R}_{\geq0}$, $e_{init} \in [0, e_{max}]$, $e_{cf} \in \mathbb{R}_{>0}$, $e_{ct} \in \mathbb{R}_{>0}$, $r \in [0, 1]$, $e_r \in \mathbb{R}_{\geq0}$, $v \in \mathbb{R}_{\geq0}$, $E_{max} \in \mathbb{R}_{\geq0}$, $E_{init} \in [0, E_{max}]$, $E_{c} \in \mathbb{R}_{>0}$, $T \in \mathbb{R}_{\geq0}$ and $k \in \mathbb{N}_{\geq0}$.

\noindent
\textbf{QUESTION:} 
One UAV, has energy capacity of $e_{max}$ and initial energy of $e_{init}$. UAV consumes energy of $e_{cf}$ per second for flight, and energy of $e_{ct}$ per second for transferring energy from UAV to the nodes in the network. The transfer efficiency rate is $r$ (i.e. node can receive energy of $e_{ct}*r$ per second). When UAV is at UAV base station, UAV can receive energy of $e_r$ per second. The moving speed of UAV is $v$. The initial position of UAV is $base$. UAV has four types of instructions: (1)hover at current position for time $t$, $t \in \mathbb{R}_{>0}$; (2)fly from $A$ to $B$, $A,B \in V$, and this instruction is invalid if UAV is not at $A$. (3)charge $A$ for $t$ second, $A \in V$ and $A$ is not $base$, $t \in \mathbb{R}_{>0}$, and this instruction is invalid if UAV is not at $A$; (4)charge UAV for $t$ second, $t \in \mathbb{R}_{>0}$, and this instruction is invalid if UAV is not at $base$.

A set of wireless sensor nodes $V' = V - {base}$. Each node has energy capacity of $E_{max}$ and initial energy of $E_{init}$. Each node consumes energy of $E_{c}$ per second.

We say the system is dead when any of the conditions is satisfied:
(1)the energy of the UAV falls to 0 and UAV is not at $base$;
(2)the energy of any of the wireless sensor nodes falls to 0.

we count it as a UAV task when UAV flies out from $base$.

The question is: is there a finite sequence of valid UAV instructions can keep the system alive for time period of $T$ with at most $k$ UAV tasks?

we call this problem as {UAVWW} problem.

\section{Proof}
It is easy to see that {UAVWW} $\in NP$, because a nondeterministic algorithm need only guess a sequence of UAV instructions and simulate the instructions in polynomial time, and check that the system is alive during the simulation, the time period passed $T$ and the number of UAV tasks is less or equal than $k$.

Given a sequence of UAV actions $S$, we can simulate the actions and verify it. The time complexity is $O(|S| * |N|)$.

Second, given an instance of the decision version of $Travelling Salesman Problem$, $G = (V, E)$ and $L$, we can transform it into an instance of $UAV Charging Problem$ in polynomial time.

We can construct $G_s = G$, and then randomly choose a vertex as the UAV base station. All the other $|V|-1$ vertices are wireless sensor network nodes.

We set the parameters of nodes as: $E_{max} = |V| + |L| - 1$, $E_{init} = |V| + |L| - 2$, $E_{min} = 0$, and $E_{c} = 1$.

We set the parameters of UAV as: $e_{max} = |V| + |L| - 1$, $e_{init} = |V| + |L| - 1$, $e_{cf} = 1$, $e_{ct} = 1$, $r = 1.0$, $e_r = 0$, and $v = 1$.

Then we ask, is the UAV able to keep the system working as long as $|V| + |L| - 1$ with at most $1$ task? Because every node need to be charged one second, and there are $|V|-1$ nodes, the answer is yes if and only if the UAV is able to find a path of length less than $|L|$ to visit all the nodes and come back to the station.

Also, it is clear that this transformation can be finished in polynomial time.
\end{document}