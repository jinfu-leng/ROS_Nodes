\documentclass[11pt]{article}
\usepackage{amssymb,amsmath}
\usepackage{enumitem}
\usepackage{graphicx}

\begin{document}
\title{UAV Wireless Power Transfer for Wireless Sensor Nodes}
\author{Jinfu Leng}
\maketitle
\section{Algorithms}
Three types of algorithms are used. The common rules of these three algorithms are:
\begin{itemize}
\item UAV gives up current work and flies back to the base station directly if its energy is insufficient to fly back to the base station after next second. This rule dominates all the other rules.
\item When UAV starts to charge a node, UAV charges the node until the node is full.
\item There is a threshold. When UAV is at base station and the min($e_{current}/e_{capacity}$) is less than this threshold, UAV goes out of base station and starts a new task.
\end{itemize}
The description of the three algorithms:
\begin{itemize}
\item LEAST POWER: This is a greedy algorithm. When UAV is out to charge sensor nodes, it always choose to charge the node with least power.
\item LEAST POWER K: Before UAV is out to charge sensor nodes, it find the top k, such as 5, nodes with least power as candidates nodes, and optimize the path which visits all these nodes in shortest path. After UAV has charged all these k nodes, UAV works as LEAST POWER.
\item LEAST POWER PARTITION: Nodes are geographically divided into four groups with equal size. UAV works like LEAST POWER, but only charge nodes which are in the same group of the node charged firstly during one UAV task.
\end{itemize}

\section{Simulation}
The distribution of nodes and the UAV:
\begin{itemize}
\item The UAV is in the center of the ground
\item The sensor nodes are randomly distributed
\end{itemize}

There are two types of node network:
\begin{itemize}
\item All the nodes have the same initial energy, energy capacity and consumption speed.
\item All the nodes have the same energy capacity and consumption speed. Initial energy are different.
\end{itemize}

The time limit of the system is 7 days.

Different size of ground, different number of sensor nodes and different thresholds are tested:
\begin{itemize}
\item Size: 10 100 500 1000
\item Number: 10 20 50 100
\item Threshold: 0.3 0.4 0.5 0.6 0.7 0.8
\end{itemize}

\section{Experiment Results}
\begin{figure}[ht!]
\centering
\includegraphics[width=15cm]{least_power_homogeneous.jpg}
\caption{Least Power for Homogeneous Network}
\label{fig:least_power_homogeneous}
\end{figure}

\begin{figure}[ht!]
\centering
\includegraphics[width=15cm]{least_power_homogeneous2.jpg}
\caption{Least Power for Homogeneous2 Network}
\label{fig:least_power_homogeneous2}
\end{figure}

\begin{enumerate}
\item The number of tasks roughly increases with the increasing of threshold, and this increasing becomes more significant when the size of the network are relatively big.
\item The system are more likely to fail with smaller threshold when the network are relatively big
\item The threshold with the value between 0.3 to 0.5 is relatively good overall.
\end{enumerate}

\begin{figure}[ht!]
\centering
\includegraphics[width=15cm]{threshold_4_homogeneous.jpg}
\caption{Threshold 0.4 for Homogeneous Network}
\label{fig:threshold_4_homogeneous}
\end{figure}

\begin{figure}[ht!]
\centering
\includegraphics[width=15cm]{threshold_4_homogeneous2.jpg}
\caption{Threshold 0.4 for Homogeneous2 Network}
\label{fig:threshold_4_homogeneous2}
\end{figure}

\begin{enumerate}
\item Three algorithms have very similar performance.
\item In network of relatively big size, LEAST POWER PARTITION and LEAST POWER K tend to have less number of tasks
\item In network of very big size, LEAST POWER is more likely to keep the system alive.

\end{enumerate}

\end{document}